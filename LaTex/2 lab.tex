\documentclass[a4paper,12pt]{article}
\usepackage[T2A]{fontenc}
\usepackage[utf8]{inputenc}
\usepackage[russian]{babel}
\usepackage{graphicx}
\usepackage{amsmath}
\usepackage{geometry}
\geometry{left=2.5cm,right=2.5cm,top=2cm,bottom=2cm}

\title{ИССЛЕДОВАНИЕ МЕЖДУНАРОДНОЙ МИГРАЦИИ РАБОЧЕЙ СИЛЫ ИЗ РОССИИ В ЗАРУБЕЖНУЮ ЕВРОПУ}
\author{А.И. Комяженкова} 
\date{Россия, ЮСНИШ «Математическое моделирование в экономике» \\
ФГБОУ ВО «ОГУ им. И.С. Тургенева»; \\
МБОУ - лицей №1 им. М. В. Ломоносова г. Орла \\
Научные руководители: \\
Е.В. Лебедева, к.п.н, доцент кафедры алгебры и математических методов в экономике, ФГБОУ ВО «ОГУ им. И.С. Тургенева»; \\
Д.Е. Ломакин, к.ф.-м.н, доцент кафедры алгебры и математических методов в экономике, ФГБОУ ВО «ОГУ им. И.С. Тургенева»}
\begin{document}
\maketitle
\section*{}
«Международная трудовая миграция стала неотъемлемой частью современной системы мирового хозяйства, фактором мирового развития, нормой существования большинства государств, обеспечивающей гибкость международного рынка труда, более рациональное использование трудовых ресурсов, взаимодействие и взаимообогащение мировых цивилизаций, приобщение развивающихся стран к мировой культуре производства, достижениям научно-технического и общественного прогресса» 

«Перемещение рабочей силы в настоящее время затрагивает интересы не только отдельных стран, но и целых регионов. По различным оценкам в настоящее время более 200 млн. человек находятся вне стран своего происхождения» 

Российские власти с недавних пор стали уделять особое внимание развитию миграционной политики и миграционного законодательства России. Становятся первоочередными задачи по удержанию россиян и дальнейшему трудоустройству в стране, к развитию системы репатриации и реинвестирования средств российских эмигрантов на родине в связи, с чем исследование является актуальным. Значимость работы состоит в том, миграция российских граждан за рубеж требует к себе пристального внимания Российских властей, поэтому полученные результаты нашего исследования, возможно, смогут своими выводами улучшить сложившуюся миграционную ситуацию, связанную с рабочей силой.

Цель исследовательской работы – определить тенденции трудовой миграции из России в страны Европы.

Достижение поставленной цели потребовало постановки и решения следующих задач:
\begin{itemize}
    \item определить масштабы трудовой эмиграции из России;
    \item провести экономико-статистический анализ процессов международной миграции рабочей силы из России;
    \item на основе результатов исследования построить прогноз численности трудовых эмигрантов.
\end{itemize}
Динамика миграции рабочей силы из Российской Федерации в зарубежную Европу с 2006 г. по 2017 г. представлена на рисунке 1.

\begin{figure}
    \centering
    \includegraphics[width=0.8\linewidth]{444.png}
    \caption{Динамика российских эмигрантов в Европу за 2006-2017гг}
    \label{fig:enter-label}
\end{figure}

Визуально просматривается снижение потока эмигрантов в европейские страны в рассматриваемом временном периоде с тенденцией близкой к линейной.
\subsection*{}
Построим линейную модель. Рассчитаем параметры уравнения регрессии:
\[
y_x = a + b \cdot t,
\]
по методу наименьших квадратов, минимизируя сумму квадратов отклонений статистических данных \( y_i \) от расчетных значений \( y_{\text{расч}} \).

Построим линейную модель. Рассчитаем параметры уравнения
регрессии:$y_x=a+b*t$,по методу наименьших квадратов, минимизируя сумму
квадратов отклонений статистических данных $y_i$ от расчетных значений $y$:


Для определения параметров $a$ и $b$ воспользуемся системой уравнений:
\[
\begin{cases}
    b\sum t^2_i = a *\sum t_i = \sum y_i*t_i, \\
    b\sum t_i = a* n + b= \sum y_i.
\end{cases}
\]

В результате расчетов (см. Таблицу 1) получена система уравнений:
\[
\begin{cases}
    555798 = 12a + 78b, \\
    2507685 = 78a + 650b.
\end{cases}
\]

\begin{table}[h]
    \centering
    \caption{Расчетная таблица для линейной модели}
    \begin{tabular}{|c|c|c|c|c|}
        \hline
        № п.п. & \( t \) & \( t^2 \) & \( t \cdot y \) & \( y \) \\
        \hline
        1 & 1 & 1 & 96212 & 96212 \\
        2 & 2 & 4 & 168862 & 84431 \\
        3 & 3 & 9 & 231297 & 77099 \\
        4 & 4 & 16 & 262648 & 65662 \\
        5 & 5 & 25 & 278310 & 55662 \\
        6 & 6 & 36 & 268734 & 44789 \\
        7 & 7 & 49 & 218071 & 31153 \\
        8 & 8 & 64 & 216984 & 27123 \\
        9 & 9 & 81 & 201420 & 22380 \\
        10 & 10 & 100 & 165780 & 16578 \\
        11 & 11 & 121 & 188551 & 17141 \\
        12 & 12 & 144 & 210816 & 17568 \\
        \hline
        \(\sum\) & \sum t=78 & \sum t^2=650 & 2507685 & 555798 \\
        \hline
    \end{tabular}
    \label{tab:table1}
\end{table}

В результате решения системы уравнений получены значения:
\[
a = 96543 \text{ чел.}, \quad b = -7727 \text{ чел./год}.
\]

\newpage
 На рисунке 2 представлены результаты аппроксимации динамики миграции рабочей силы из России в зарубежную Европу линейной трендовой моделью.



Математическая модель тренда выражается следующей формулой:
\[
y_x = 96543,86 - 7727,29 \cdot t.
\]

\begin{figure}
    \centering
    \includegraphics[width=0.5\linewidth]{333.png}
    \caption{Аппроксимация эмигрантов из России в европейские страны линейной модели}
    \label{fig:enter-label}
\end{figure}

Коэффициенты интерпретируются следующим образом:
\begin{itemize}
    \item \( a = 96543 \) чел. представляет собой расчетное начальное значение за 2006 г. числа выбывших из РФ в страны Европы.
    \item \( b = -7727 \) чел./год означает расчетный годовой абсолютный прирост числа выбывших.
\end{itemize}

Относительная точность полученной модели составила 10,8\%. Точность модели считается удовлетворительной, если средняя относительная погрешность меньше 15\%, таким образом, модель можно считать приемлемой и использовать для построения прогноза.

С учетом того, что ряд динамики содержит показатели с 2006 по 2017 гг., прогноз можно дать только на следующие три года. При этом следует учитывать, что чем больше глубина прогноза, тем меньше точность прогноза. Результаты прогноза говорят о дальнейшей тенденции увеличения миграции рабочей силы в европейские страны.

Проведенное исследование позволило выявить тенденции трудовой миграции из России в страны Европы. Построенная линейная модель показала удовлетворительную точность и может быть использована для прогнозирования. Полученные результаты могут быть полезны для разработки мер по улучшению миграционной политики России.

\begin{thebibliography}{9}
    \bibitem{Aliev2011} 
    Алиев, М.Д. Россия в международных миграционных процессах: Автореф. дис.канд.экон.наук: 08.00.14 – Мировая экономика / М.Д. Алиев С.-Петерб.гос.ун-т. – СПб., 2011г.– 26с.
    
    \bibitem{StudBooks} 
    Трудовые права граждан России за рубежом [Электронный ресурс]. URL: \texttt{https://studbooks.net/1058812/pravo/trudovye\_prava\_grazhdan\_rossii\_za\_rubezhom} (дата обращения 31.01.2019).
\end{thebibliography}

\end{document}