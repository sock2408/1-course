\documentclass[12pt]{beamer}
\usepackage[T2A]{fontenc}
\usepackage[utf8]{inputenc}
\usepackage[russian]{babel}
\usepackage{graphicx}
\usepackage{tikz}
\usetikzlibrary{positioning, shadows}
\usepackage{booktabs}
\usepackage{mwe}
\setbeamertemplate{background}{
    \includegraphics[width=\paperwidth,height=\paperheight]{parallax-bg.jpg}
}

\usetheme{Dresden}
\usecolortheme{seahorse}
\setbeamertemplate{itemize items}[triangle]
\setbeamercolor{title}{fg=blue!80!black}
\setbeamercolor{frametitle}{fg=blue!70!black}

\title{История физико-математического факультета}
\subtitle{Орловского государственного университета}
\author{Даниил Баранов}
\date{}

\begin{document}

\begin{frame}[plain]
    \begin{tikzpicture}[remember picture,overlay]
        \node[opacity=0.3] at (current page.center) {\includegraphics[width=\paperwidth]{example-image-a}};
        \fill[blue!80!black, opacity=0.7] (current page.south west) rectangle ([yshift=3cm]current page.north east);
        \node[text width=0.8\paperwidth, text=white, font=\Huge\bfseries, align=center] at (current page.center) {
            История физмата\\1931–2023
        };
    \end{tikzpicture}
\end{frame}

\begin{frame}{Рождение факультета (1931-1941)}
    \begin{columns}
        \column{0.5\textwidth}
        \includegraphics[width=\textwidth]{}

\begin{figure}
            \centering
            \includegraphics[width=1\linewidth]{}
\begin{figure}
                \centering
                \includegraphics[width=1\linewidth]{image.png}
                \caption{Первое здание института}
                \label{fig:enter-label}
            \end{figure}
            \label{fig:enter-label}
        \end{figure}

        \column{0.5\textwidth}
        \begin{itemize}
            \item 1931: Создание индустриально-педагогического института
            \item 121 студент, 11 преподавателей
            \item 1932: Преобразование в пединститут
            \item 1935: Первый выпуск - 45 специалистов
        \end{itemize}
    \end{columns}
\end{frame}

\begin{frame}{Военные годы и подвиги}
    \centering
    \begin{tikzpicture}
        \node[drop shadow] (war) {\includegraphics[height=4cm]{615-193x300.jpg}};
        \node[below=0.1cm of war, text width=0.8\textwidth, align=center] {
            24 преподавателя и студента погибли на войне\\
            \small{Среди них - Герой Советского Союза В.А. Бульчев}
        };
    \end{tikzpicture}
    
    \vspace{0.5cm}
    \begin{alertblock}{Память}
        Студент Дмитрий Софий (расстрелян в 1942) посмертно награжден медалью
    \end{alertblock}
\end{frame}

\begin{frame}{1950-е: Восстановление и развитие}
    \begin{table}
        \centering
        \begin{tabular}{lc}
            \toprule
            Год & Успеваемость \\
            \midrule
            1935 & 87.7\% \\
            1936 & 93.5\% \\
            1956-1957 & 95\% \\
            \bottomrule
        \end{tabular}
        \caption{Динамика успеваемости студентов}
    \end{table}
    
    \begin{columns}
        \column{0.5\textwidth}
        \begin{itemize}
            \item Сталинские стипендии
            \item Новое здание (1957)
        \end{itemize}
        
        \column{0.5\textwidth}
        \includegraphics[width=\textwidth]{1311-300x175.jpg}
    \end{columns}
\end{frame}

\begin{frame}{Научные достижения}
    \centering
    \begin{tikzpicture}
        \node (science) {\includegraphics[height=3cm]{812-220x300.jpg}};
        
        \node[below=0.1cm of science] {3-томная "История физики"};
    \end{tikzpicture}
    
    \vspace{0.5cm}
    \begin{exampleblock}{Известные выпускники}
        Игорь Струков - разработчик космической платформы для изучения "черных дыр"
    \end{exampleblock}
\end{frame}

\begin{frame}{Современный физмат}
    \begin{columns}
        \column{0.4\textwidth}
        \includegraphics[width=\textwidth]{282-548x768.jpg}
        
        \column{0.6\textwidth}
        \begin{itemize}
            \item 7 кафедр
            \item 634 студента (2007)
            \item 8 докторов наук
            \item Компьютерные классы
            \item Спортивные традиции
        \end{itemize}
    \end{columns}
    
    \vspace{0.5cm}
    \centering
    \begin{block}{}
        "Физмат был, есть и остаётся основой университета!"
    \end{block}
\end{frame}

\end{document}