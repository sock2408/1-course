\documentclass{article}
\usepackage[russian]{babel}

\begin{document}

\section*{Страница 1}

102 УРАВНЕНИЯ ГИПЕРБОЛИЧЕСКОГО ТИПА [ГЛ. II]

Переходя к пределу при $\varepsilon \to 0$, видим, что существует предел

\[
\lim_{\varepsilon \to 0} u_\varepsilon (t) = \lim_{\varepsilon \to 0} U \left( t - \tau_\varepsilon \right) = U(t),
\]

что и доказывает наше утверждение.

Перейдем к представлению решения неоднородного уравнения через $U(t)$ — функцию влияния мгновенного импульса. Разбивая промежуток $(0, t)$ точками $\tau_i$ на равные части

\[
\Delta \tau = \frac{t}{m},
\]

представим функцию $f(t)$ в виде

\[
f(t) = \sum_{i=1}^m f_i(t),
\]

где

\[
f_i(t) = 
\begin{cases} 
0 & \text{при } t < \tau_i \text{ и } t \geq \tau_{i+1}, \\
f(t) & \text{при } \tau_i \leq t < \tau_{i+1}.
\end{cases}
\]

Тогда функция

\[
u(t) = \sum_{i=1}^m u_i(t),
\]

где $u_i(t)$ суть решения уравнения $L(u_i) = f_i$ с нулевыми начальными данными.

Если $m$ достаточно велико, то функцию $u_i(t)$ можно рассматривать как функцию влияния мгновенного импульса интенсивности

\[
I = f_i(\tau_i) \Delta \tau = f(\tau_i) \Delta \tau,
\]

так что

\[
u(t) = \sum_{i=1}^m U \left( t - \tau_i \right) f(\tau_i) \Delta \tau \cdot \frac{\Delta \tau}{\Delta \tau} \int_0^t U \left( t - \tau \right) f(\tau) d\tau,
\]

т. е. мы приходим к формуле

\[
u(t) = \int_0^t U \left( t - \tau \right) f(\tau) d\tau,
\]

показывающей, что влияние непрерывно действующей силы можно представлять суперпозицией влияний мгновенных импульсов.

В рассмотренном выше случае $u^{(1)}$ удовлетворяет уравнению (50) и условиям $u_n(0) = \dot{u}_n(0) = 0$. Для функции влияния $U(t)$ имеем:

\[
\ddot{U} + \left( \frac{\pi n}{l} \right)^2 a^2 U = 0, \quad U(0) = 0, \quad \dot{U}(0) = 1,
\]

так что

\[
U(t) = \frac{l}{\pi n a} \sin \frac{\pi n}{l} a t.
\]

Отсюда и из (3*) получаем формулу (52)

\[
u_n^{(1)} (t) = \int_0^t U \left( t - \tau \right) f_n (\tau) d\tau = \frac{l}{\pi n a} \int_0^t \sin \frac{\pi n}{l} a \left( t - \tau \right) f_n (\tau) d\tau.
\]

\section*{Страница 2}

МЕТОД РАЗДЕЛЕНИЯ ПЕРЕМЕННЫХ 103

Полученное выше интегральное представление (3*) решения обыкновенного дифференциального уравнения (1*) имеет, как мы убедились, тот же физический смысл, что и формула (59), дающая интегральное представление решения неоднородного уравнения колебаний.

5. Общая первая краевая задача. Рассмотрим о б щ у ю п е р в у ю к р а е в у ю з а д а ч у для уравнения колебаний:
найти решение уравнения
\[
u_{tt} = a^2 u_{xx} + f(x, t), \quad 0 < x < l, \quad t > 0
\]
с дополнительными условиями
\[
u(x, 0) = \varphi(x), \quad 0 \leq x \leq l;
\]
\[
u_t(x, 0) = \psi(x), \quad 0 \leq x \leq l;
\]
\[
u(0, t) = \mu_1(t), \quad t \geq 0;
\]
\[
u(l, t) = \mu_2(t), \quad t \geq 0.
\]

Введем новую неизвестную функцию $v(x, t)$, полагая:
\[
u(x, t) = U(x, t) + v(x, t),
\]
так что $v(x, t)$ представляет отклонение функции $u(x, t)$ от некоторой известной функции $U(x, t)$.

Эта функция $v(x, t)$ будет определяться как решение уравнения
\[
v_{tt} = a^2 v_{xx} + \tilde{f}(x, t), \quad \tilde{f}(x, t) = f(x, t) - [U_{tt} - a^2 U_{xx}]
\]
с дополнительными условиями
\[
v(x, 0) = \Phi(x), \quad \Phi(x) = \varphi(x) - U(x, 0);
\]
\[
v_t(x, 0) = \bar{\Phi}(x); \quad \bar{\Phi}(x) = \psi(x) - U_t(x, 0);
\]
\[
v(0, t) = \bar{\mu}_1(t), \quad \bar{\mu}_1(t) = \mu_1(t) - U(0, t);
\]
\[
v(l, t) = \bar{\mu}_2(t); \quad \bar{\mu}_2(t) = \mu_2(t) - U(l, t).
\]

Выберем вспомогательную функцию $U(x, t)$, таким образом, чтобы
\[
\bar{\mu}_1(t) = 0 \quad \text{и} \quad \bar{\mu}_2(t) = 0;
\]
для этого достаточно положить
\[
U(x, t) = \mu_1(t) + \frac{x}{l} [\mu_2(t) - \mu_1(t)].
\]

Тем самым общая краевая задача для функции $u(x, t)$ сведена к краевой задаче для функции $v(x, t)$ при нулевых граничных условиях. Метод решения этой задачи изложен выше (см. п. 4).

\end{document}